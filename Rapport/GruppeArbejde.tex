\subsection{Gruppe Arbejde}
Vi valgte at køre 2 ugers sprints, dette passede bedst med størrelserne af vores
PBI'er og Mødetiderne med vores PO ifht. til sprintreviews.
I begyndelsen af hver sprint brugte vi planning poker for at vurdere hvor lang tid
hver af de forskellige opgaver under sprinten vil tage. Det gjorde det nemmere at se 
om vurdere vores fremskridt i sprintet, også selvom man nogengange vurdere tidsforbruget på opgaven forkert.


Dertil hjalp det de næste sprints fremadrettet, da man så ved bedre hvor lang tid visse opgaver kræver.
Efter hver uge i projektet kørte vi en happiness index for hver person. Det havde vi internt i gruppen valgt
at bruge for have en status over hvordan folk syntes projektet gik, og om de havde evt. grunde til deres happiness.
Ens happiness, som gik på en skala 1-5, kunne være alt fra noget projektrelateret til bare hvorledes man havde det.

Under kodning gjorde vi stor brug af pairprogramming, dette hjalp med at mindske fejl, sikrerede for bedre løsninger
og gav os en større produktivitet. Ud over dette aftalte vi i gruppen at efter noget er blevet lavet, 
skal det reviewes af et andet medlem i gruppen, før det kan kvalificeres som færdigt.