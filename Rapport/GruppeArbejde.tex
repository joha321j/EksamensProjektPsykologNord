\subsection{Gruppe Arbejde}
I denne del af rapporten vil vi beskrive hvordan vi har arbejdet som en gruppe, samt hvad det har bidraget med til vores fremskridt i projektet. 
Til at strukturer vores projekttid benyter vi os af SCRUM, dette bruger vi fordi vi med SCRUM opnår hurtigere fremskridt, samtidig med at vi fanger flere fejl ved at få flere øjne på det gennem vores reviews. Vi har valgt at kører 2 ugers sprints for at det passede bedst med vores PO og hvornår de havde tid til at holde sprintreview møder, samtidig med at vi skulle have tid til at få lavet noget der var værd at vise PO.
I starten af vores Sprint 0, sad vi sammen med vores PO og fik rangeret de ting de gerne ville have lavet så vi vidste hvor vi skulle starte da vi kom til vores Sprint 1. Som det første i alle vores sprints sidder vi alle fire i gruppe og laver planing poker over vores opgaver for at få lavet et tidsestimering for at vi kan se hvor meget der skal arbejdes med i den sprint og om der måske skal tages mere med eller om vi har taget for meget med så vi ikke kan nå at komme i mål med det vi vil. Vi har også valgt at vi hver fredag i vores gruppe har et happines index hvor vi på en skala fra 1-5 bedømer hvor godt ugen er gået, 1 er lort personen er gået ned med stress, 5 er perfekt alt har bare spillet man er blevet langt hurtigere færdig med sine ting end forventet. Samtidig om fredagen har vi, for at holde en god stemning i gruppen, lavet en kageordning hvor vi skiftes til at have kage med til gruppen.
I selve arbejdsprocessen har vi kørt meget med pairprograming hvor der har sidet to personer og arbejdet sammen om den samme task, dette har virket fornuftigt da vi har fået lavet en del. Ud over at vi kører pairprogramming har vi også lagt meget vægt på at vi skal have alt forbi mere end dem der har sidet og udarbejdet det for at opdage, eventuele fejl og for at alle ved hvad der er blevet lavet. Dette har vi gjort ved at bruge branches i git og inden at man kan få noget ned på master skal der laves et pull request som skal kigges igennme af en anden i gruppen.