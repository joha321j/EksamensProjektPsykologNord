\subsection{Gruppekontrakt}


\textbf{Kontrakten er udarbejdet i forhold til de 5R'er (Rammer, Relationer, Roller og Regler) }

\textbf{SEMESTER OG HOLD} 

2.semester DK - B 

 

\textbf{NAVN PÅ PROJEKT} 

Eksamens projekt Psykolog nord 


\textbf{FOR- OG EFTERNAVN PÅ GRUPPEMEDLEMMER} 

1. Sebastian Thorup Frederiksen 

2. Anders Fredensborg Rasmussen 

3. Kaare Veggerby Sandbøl 

4. Johannes Ehlers Nyholm Thomsen 

 

\textbf{KONTAKT INFORMATION} 

1. E-mail Sthorupf@hotmail.com  Mobil nr. 22904692 

2. E-mail ande714b@edu.eal.dk  Mobil nr. 61282954 

3. E-mail kaarevs@hotmail.com  Mobil nr. 292405062 

4. E-mail Joha321j@edu.eal.dk Mobil nr. 60572253 
 

 

\textbf{HVAD ER VORES KOMPETENCER / HVAD ER VI GODE TIL? }
\begin{itemize}
\item Vær på 

\item Struktur 

\item Opsøg viden 

\item Positiv indstilling 

\item Kommunikation 

\item Pusterum 

\item Problemløsning 

\item Refleksion 
\end{itemize}
\textbf{HVOR OG HVORNÅR ARBEJDER VI?  (Mødetidspunkt og sted) }
\begin{itemize}
\item Som udgangspunkt efter skole i tiden tildelt projektarbejde - med mindre andet aftalt. 
\end{itemize}

\textbf{HVORDAN SKAL SAMARBEJDE OG KOMMUNIKATION INTERNT I GRUPPEN VÆRE? }
\begin{itemize}
\item God attitude 

\item God arbejdsmoral 

\item Positiv og åben kommunikation 
\end{itemize}

\textbf{HVORDAN SKAL SAMARBEJDE OG KOMMUNIKATION MED OMVERDEN VÆRE? }
\begin{itemize}
\item Åben for sparring med andre grupper. 
\end{itemize}
 

\textbf{HVORDAN TRÆFFES BESLUTNINGER I GRUPPEN? } 
\begin{itemize}
\item Diskussion, argumenter for sin sag. Stemme for sin sag. Demokratisk. 
\end{itemize}
 

\textbf{HVORDAN HÅNDTERES KONFLIKTER INTERNT I GRUPPEN? }
\begin{itemize}
\item Kommunikere med hinanden. Forklare konflikten og finde på løsning. 
\end{itemize}

\textbf{HVAD ER GRUPPENS MÅL / DELMÅL? (Forventningsafstemning) 
}
\begin{itemize}
\item Aflevere noget vi alle er glade for 

\item Lære at arbejde i grupper 

\item Tilegne sig ny viden
\end{itemize}

\textbf{HVAD SKAL DER TIL FOR AT NÅ MÅLENE? }
\begin{itemize}
\item Tid 

\item Fokus 

\item Høj arbejdsmoral 
\end{itemize}
 

\textbf{HVILKEN NIVEAU AF KVALITET ARBEJDES DER EFTER AT OPNÅ? }
\begin{itemize}
\item Tilfreds med egen indsats.  
\end{itemize}

\textbf{HVAD SKAL DER TIL FOR AT OPNÅ DEN ØNSKEDE KVALITET? }
\begin{itemize}
\item Tid 

\item Fokus 

\item Høj arbejdsmoral 
\end{itemize}

\textbf{HVEM ER PROJEKTLEDER I GRUPPEN? }
\begin{itemize}
\item Fælles indsats 
\end{itemize}

\textbf{HVORDAN ER SPILLEREGLERNE FOR OPRETHOLDELSE AF EN GOD GRUPPE DYNAMIC?  }

\textbf{(Ex. Vi vil have en god omgangstone, alle har initiativpligt, alle skal udvise respekt til hinanden, alle skal dele ud af sin erfaring etc.) }
\begin{itemize}
\item Erkende sin fejl.  

\item Hjælpe hinanden 

\item Stille spørgsmål 
\end{itemize}

\textbf{SAMLET OVERSIGT OVER GRUPPENS INTERNE SPILLEREGLER:}  

\textbf{(Ex. Mødetidspunkt, mødepligt, kommunikation, deling af filer/dokumenter, møder, håndtering af konflikter etc.) }

 
\begin{itemize}
\item Mødetidspunkt/-pligt: Møde medmindre andet siges, arbejde indtil enighed om at stoppe. 

\item Deling af filer/dokumenter: OneDrive, GitHub, Trello, og LucidChart drev. 

\item Håndtering af konflikter: Intern kommunikation og udredelse, i værste tilfælde tale med undervisere. 

\item Kodestandard: Vi gør brug af Microsofts Csharp kodestandard. https://docs.microsoft.com/en-us/dotnet/csharp/programming-guide/inside-a-program/coding-conventions  

\item Fredagskage: 

Anders 

Johannes 

Sebastian 

Kaare 
\end{itemize}
\textbf{HVAD ER KONSEKVENSEN AF BRUD PÅ GRUPPENS INTERNE SPILLEREGLER? }
\begin{itemize}
\item snak med undervisere. 

\item Kage. 
\end{itemize}
