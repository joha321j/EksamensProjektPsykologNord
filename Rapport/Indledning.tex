\section{Indledning}
Vores PO var PsykologNord, som er Danmarks største psykolog kæde, og de har deres hovedsæde i Aalborg.
De slår sig op på at klienter kan komme til en konsultation med en psykolog inden for 24 timer.
På nuværende tidspunkt bruger de Terapeutbooking, til at styre deres bookinger og det meste andet, der har noget at gøre med den daglige drift at klinikerne.
Vores opgave blev at lave en erstatning til Terapeutbooking, da det er målrettet til enkeltmandsvirksomheder, og det derfor ikke kan håndtere flere behandlere, hvilket er nødvendigt når man har en kæde af kliniker.
Så vores største opgave var at lave et booking system hvor behandler og lokaler ikke kunne dobbelt bookes, hvilket er PsykologNords største problem med deres nuværende løsning.

Rapporten vil arbejde med følgende: En analyse af PsykologNords nuværende situation.
En modellering af deres virksomhed og problemet.
En modellering af den ønskede løsning og hvordan løsningen er blevet implementeret.
Hvad vi tænker, der skal laves af videre arbejde, og hvordan vi forventer det skulle gøres.
Og til sidst hvordan arbejdet har fungeret i gruppen, og hvad vi har lært.

Alt i alt har projektet været meget lærerigt for alle gruppemedlemmer.
Det har været meget givende at have et længere forløb, hvor vi har kunnet fordybe os i et firma og deres problemer.

\newpage