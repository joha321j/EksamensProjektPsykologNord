\subsection{Forretningsmodel}
\label{section:forretningsmodel}

PsykologNord tilbyder psykologbehandling uden ventetid.
Derfor har de åbent alle 7 dage i ugen fra kl. 9 til 21. Det eneste krav er at du ikke kan ændre en tid inden for 24 timer af aftalens start.

PsykologNord er en psykologkæde med 3 afdelinger:
\begin{itemize}
    \item Aalborg
    \item Aarhus
    \item Odense
\end{itemize}

De har 3 psykologer:

\begin{itemize}
    \item Katrine Breum Larsen
    \item Miriam Poulsen
    \item Stefan Guldager Boldemann
\end{itemize}

Lige nu er Katrine og Miriam begge tilknyttet afdelingerne i Aalborg og Aarhus, og Stefan er tilknyttet afdelingen i Odense.


Firmaet drives af Lasse Kirk, og hans kæreste Katrine Breum Larsen.
De har en flad firmastruktur, begge indgår i firmaets ledelse. Lasse ejer firmaet, hvor Katrine står for den daglige drift. 
Firmaet tilhører den basale form fra Mintzbergs fem organisationsformer, og på 
figur \ref{forretning:organisationsdiagram} kan man se deres virksomhedsstruktur.

De tilknyttede psykologer er ikke ansat af PsykologNord, men er partnere.
Derfor skal de ikke betale løn og pension, men derimod har de aftalt, at psykologen modtager halvdelen af betalingen fra klienten, og PsykologNord modtager den anden halvdel.

De bruger marketingsfirmaet Bundtofte, som holder styr på deres hjemmeside. De holder også deres socialie medier updateret; de er på Facebook og Instagram.

Deres kontakt med klienter er personlig, da du har en tilknyttet psykolog, som 
du har samtaler med. En oversigt over deres forretningsmodel kan ses på figur \ref{forretning:fml}.

\begin{figure}[t]
    \caption{Organisationsdiagram for PsykologNord}
    \centering
        \includegraphics[scale=.8]{OrganisationsDiagram}
    \label{forretning:organisationsdiagram}
\end{figure}

\begin{figure}[b]
    \caption{FML for PsykologNord}
    \centering
        \includegraphics[width=\textwidth]{FML}
    \label{forretning:fml}
\end{figure}

