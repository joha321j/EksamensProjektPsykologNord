%\textbf{Operation:} Opret aftale 
%
%\textbf{Cross reference:} Opret aftale 
%
%\textbf{Precondition:} Der skal være mindst et brugerobject og der skal være mindst et klientobject I systemet, der skal også være en klinik, begge aktører skal være ledige på det valgte tidspunkt, der skal samtidig også være et lokale ledigt I den valgte klinik på det valgte tidspunkt. 
%
%\textbf{Postconditions:}   
 
Med alle use casene modelleret kan vi går videre til at modellere hvad koden skal gøre. Hertil har vi så oprettet SOC'er for de mest relevante metoder, så vi ved præcis hvordan funktionen skal virke.


\subsection{Systemoperationskontrakter(SOC) for Use Case: Book ny aftale}\label{Operation:Book ny aftale}

{\setlength{\parindent}{0cm}
\textbf{Operation:} createClient(clientName: string, clientEmail: string, clientPhoneNumber: string, clientAddress: string, clientSocialSecurityNumber: integer, clientNote: string) 

\textbf{Cross References:} Use Case: Book ny aftale 

\textbf{Precondition:} Client instance with Client.Email equals to clientEmail does not exist in the system

\textbf{Postcondition:}  
		\begin{itemize}
			\item Client instance c was created 
			\item Attributes of c were initialized
		\end{itemize}
 


\textbf{Operation:} createAppointment(dateAndTime: DateTime, users : User[2..*], appointmentType: AppointmentType, note: string, room: Room) 

\textbf{Cross References:} Use Case: Book ny aftale 

\textbf{Precondition: }
		\begin{itemize}
			\item Client instance c exists 
			\item User instance u exists 
			\item room instance r exists 
			\item r has no appointment at dateAndTime
		\end{itemize}
		
\textbf{Postcondition:}  
        \begin{itemize}
            \item An appointment instance a is created
            \item Attributes of a were intialized
            \item a was associated with c, u and r
        \end{itemize}
        
{\setlength{\parindent}{14pt}