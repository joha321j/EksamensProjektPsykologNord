%\textbf{Operation:} Opret aftale 
%
%\textbf{Cross reference:} Opret aftale 
%
%\textbf{Precondition:} Der skal være mindst et brugerobject og der skal være mindst et klientobject I systemet, der skal også være en klinik, begge aktører skal være ledige på det valgte tidspunkt, der skal samtidig også være et lokale ledigt I den valgte klinik på det valgte tidspunkt. 
%
%\textbf{Postconditions:}   

\subsection{Systemoperationskontrakter(SOC) for Use Case: Book ny aftale}\label{Operation:Book ny aftale}

Ud fra et SSD kan man lave SOC'er, der forklarer, hvad der skal ske af tilstandsændringer i systemet, når en bruger laver en systemoperation.
En systemoperation er det, der repræsenteres i et SSD som en pil fra brugeren til systemet.
Det vigtige med en SOC er at beskrive hvilke ændringer, der er sket, ikke hvordan de er sket, det er et sekvensdiagram, der beskriver det. For at se SOC'en omkring createClient se bilag \ref{bilag:SOCcreateClient} på side \pageref{bilag:SOCcreateClient}



\textbf{Operation:} createAppointment(dateAndTime: DateTime, users : User[2..*], appointmentType: AppointmentType, note: string, room: Room, notificationTime: TimeSpan, emailNotification: bool, smsNotification: bool) 

\textbf{Cross References:} Use Case: Book ny aftale 

\textbf{Precondition: }
		\begin{itemize}
			\item Client instance c exists 
			\item User instance u exists 
			\item room instance r exists 
			\item r has no appointment at dateAndTime
		\end{itemize}
		
\textbf{Postcondition:}  
        \begin{itemize}
            \item An appointment instance a is created
            \item Attributes of a were intialized
            \item a was associated with c, u and r
        \end{itemize}
        
\textbf{Output:}
There is sent a confirmation email and/or SMS to the client
{\setlength{\parindent}{14pt}