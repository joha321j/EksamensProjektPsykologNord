\subsection{Use Cases}
\label{UseCases}
For at sikre at vi laver et program, der fungerer som ønsket, har vi opstillet flere use cases for systemet.
Disse use cases er blevet udledt på baggrund af møder med PsykologNords ledelse og ved brug af de fire basisfunktioner for persistens: create, read, update og delete(CRUD) på objekt- og domænemodelen.
Derudover har vi også identificeret primæraktørene og deres mål, og sikret at vores use cases opfylder målene.

Vores primæraktører og deres mål er:

\begin{itemize}
    \item Bookingansvarlig
        \begin{itemize}
            \item Se aftaler
            \item Book aftale
            \item Ændr aftale
            \item Aflys aftale
        \end{itemize}
    \item Psykolog
        \begin{itemize}
            \item Ajourfør kundes journal
        \end{itemize}
    \item Kunde
        \begin{itemize}
            \item Se og betal fakturaer
            \item Book aftale
            \item Ændr aftale
            \item Aflys aftale
        \end{itemize}

\end{itemize}

Det er f.eks. vigtigt at skelne mellem at booke en ny aftale, ændre den og aflyse den, da det er tre forskellige funktionaliteter i systemet.

\begin{figure}[ht]
	\centering
  		\includegraphics[scale=0.75]{UseCaseDiagram.png}
  \caption{Use Case diagram.}
  \label{fig:UseCaseDiagram}
\end{figure}

På figur \ref{fig:UseCaseDiagram} kan man se alle de use cases, vi fandt frem til.
Uses casesne er blevet markeret med farver, der repræsenterer deres prioritering under projektforløbet: de grønne var dem, som PsykologNord først ønskede implementeret og derefter de gule use cases.
I følgende afsnit ses en use case beskrevet mere detaljeret. 

To personer deltager i en aftale: en kunde og en psykolog.

\subsubsection{Use Case: Book aftale}\label{usecase:bookaftale}
{\setlength{\parindent}{0cm}
\textbf{Scope:} Bookingsystem for PsykologNord

\textbf{Primær Aktør:} Bookingansvarlig

\textbf{Hovedscenarie (succes):} Booking af ny aftale

Den bookingansvarlige ønsker at booke en aftale.
Hvis den involverede kunde ikke har brugt PsykologNord før, oprettes kunden i systemet.
Kunden vælger aftaletype og der bookes en ny aftale, hvor både kunden og psykologen har tid. 
Aftalen bookes og kunde og psykolog notificeres.

\textbf{Primær Aktør:} Kunde og bookingansvarlig

\textbf{Alternativt scenarie (succes):} Kontakt PsykologNord for at booke en aftale

En kunde ønsker at booke en aftale og kontakter derfor PsykologNord.
En bookingansvarlig booker aftalen som i hovedscenariet.

\textbf{Primær Aktør:} Kunde

\textbf{Alternativt scenarie (succes):} Vælge parterapi som aftale

En kunde ønsker at booke en parterapiaftale.
Hvis den involverede kunde ikke har brugt PsykologNord før, oprettes kunden i systemet.
Kunden vælger aftaletype og indtaster den anden kundes informationer, og den anden kunde oprettes også i systemet.
Der bookes en ny aftale, hvor både kunden og psykologen har tid. 
Aftalen bookes, begge kunder og psykologen notificeres.
}

For at se resten af use casene se bilag \ref{bilag:UseCases}, på side \pageref{bilag:UseCases}
