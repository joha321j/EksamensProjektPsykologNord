\section{Programmering}
\subsection{Kodestandarder og TDD}
\label{Kodestandarder og TDD}

Før vi kodet en enkelt linje af kode, havde vi en kodestandard på plads. Uden en kodestandard kunne koden hurtigt gået hen og været uoverskuelig. Ligeledes er kodestandarder en effektiv måde at lave kvalitetssikring på selve koden. I vores projekt har der ikke været den største diskussion om hvad kodestandarderne skulle være. De har dog altid være brugbare i tilfælde af tvivl omkring hvorledes noget skulle navngives eller struktures. Vi gjorde brug af Microsofts C\# kodestandard, da vi hurtigt blev enige om at den gav mest mening.


Derved betød det også f.eks. at alle ting som var interne i sin klasse havde et understregningstegn før sig. Dertil aftalte vi i gruppen at vi ville kører Test-Driven Development, fremover forkortet til TDD. Årsagen til dette var pga. vores introduktion til det tidligere på studiet og en fælles interesse i at afprøve det. Det betød således at vi altid skulle opsætte tests før skrev andet kode. Det gjorde at vi fik lavet markant flere tests end ved almindelig udvikling.