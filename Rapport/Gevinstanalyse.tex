\subsection{Hvad vil NoName give Psykolog Nord?}
Dette afsnit vil beskrive, hvad formålet med projektet er, hvad Psykolog Nord vil få ud af det, og hvordan vi vil sikre gevinsterne.

\subsubsection{Formål}
Formålet med projektet er:

\begin{itemize}
    \item At forbedre booking processen, så når en klient laver en aftale med en psykolog i et lokale, kan en anden psykolog ikke lave en aftale på samme tidspunkt i samme lokale, uden at psykologen manuelt skal ind og blokere for booking af lokalet.
    \item At forbedre booking processen, så når en klient laver en aftale med en psykolog i en afdeling, kan en anden klient ikke lave en aftale med samme psykolog i en anden afdeling på samme tidspunkt, uden at psykologen manuelt skal melde sig optaget.
    \item At forbedre kundesikkerheden ved at sikre, at en psykolog kun har adgang til sine egne klienters journaler.
\end{itemize}

\subsubsection{Forretningsmæssig løsningsbeskrivelse}

Projektet skal sikre, at Psykolog Nords bookingproces kommer til at foregå uden, at de manuelt skal ind og rette i hinandens kalendere.
Derudover skal det sikre, at deres psykologer kun kan se deres egne klienters journaler. 
Til sidst skal det også give mulighed for at lave brugere af systemet med andre rettigheder, såsom en sekretær, der vil kunne se alle klienter og deres fakturaer, men ikke kunne se nogen klientjournaler.

Til at starte med skal løsningen bruges af Psykolog Nord, men på længere sigt skal løsningen også bruges af Katrine, Lasse og Anders's andre firmaer.

De ønsker at tage løsningen i brug hurtigst muligt.

\subsubsection{IT-mæssig løsningsbeskrivelse}

Psykolog Nord vil i fremtiden gerne have bookingsystemet til at være en hjemmeside, men det er på nuværende tidspunkt uden for projektgruppens pensum.
Derfor er de gået med til, at det bliver udviklet som en WPF applikation i .Net C\#, hvor applikationen skal kunne køre på flere computere i et netværk.

Psykolog Nord har også mange forslag til videreudvikling af systemet, som vi ved, at vi ikke vil kunne nå inden for tidsrammen af projektet.
Derfor er det vigtigt, at løsningen er objektorienteret, så den er nem at videreudvikle på et senere tidspunkt.
Da vi også nu ved, at de senere vil have det som en hjemmeside i stedet for en applikation, skal løsningen også være lagdelt, så blandt andet brugergrænsefladen nemt kan udskiftes.

\subsubsection{Effekter af projektet}
\paragraph*{Økonomiske effekter}
\subparagraph*{Forretningsmæssige investeringer}

Da løsningen skal ind og erstatte en allerede eksisterende løsning skal der stræbes efter, at der ikke bliver den store omvæltning for Psykolog Nord.

\subparagraph{IT investeringer}

\textit{Interne ressourcer:} Psykolog Nord skal bruge tid på møder med projektgruppen. Der skal også bruges tid på at svare på spørgsmål over mail og telefon.

\textit{Eksterne ressourcer:} Projektgruppen har startet projektet, men vi har ikke nået at færdiggøre løsningen med alle ønskerne Psykolog Nord har.
Derfor vil der bagefter skulle hyres et software firma til at videreudvikle på løsningen.

Der skal ikke investeres i hardware, da deres nuværende løsning allerede er digital. Derfor er den totale udgift prisen på at få færdig udviklet løsningen.

\subparagraph{Økonomiske gevinster}

\textit{Forretningsmæssige gevinster:} Det forventes, at den gennemsnitlige tid på at booke en aftale vil blive mindsket.
Der vil også blive en større sikkerhed for klienterne, da alle brugere af systemet ikke vil have adgang til alle klienters oplysninger.

I det første år eller 2 vil projektet være en udgift, men effektiviseringen af bookingprocessen vil resultere i en besparelse på længere sigt.

\subsubsection{Risikovurdering}

\paragraph*{Læk af følsomme personoplysninger}

\subparagraph{Risikovurdering:}
Høj pga. økonomiske konsekvenser.
    
\subparagraph{Eksisterende sikring:}
Den nuværende løsning lever op  til alle krav fra Datatilsynet mht. til behandling af følsomme personoplysninger, blandt andet ved at have en krypteret hjemmside.\cite{terapeutbookingsikkerhed}
    
\subparagraph{Handlingsplan:}
Den nye løsning skal også leve op til reglerne fra Datatilsynet. Derudover vil den forbedre sikkerheden ved at begrænse brugernes adgang til klienter i systemet, så en bruger kun har adgang til sine egne klienter.
    
Hvis der skulle ske en læk af data vil Psykolog følge lovgivningen ved at give Datatilsynet besked, inden for 24 timer efter lækken opdages, og derefter give de involverede klienter besked. Derefter vil lækken lukkes hurtigst muligt.
    
\paragraph*{Tab af klienters fortrolige oplysninger som følge af servernedbrud}

\subparagraph{Risikovurdering:}
Lav
    
\subparagraph{Eksisterende sikring:}
Den nuværende serverløsning har en gennemsnitlig oppetid på 99.5\%. De har 2 separate strømtilslutning, hvilket sikrer deres kørsel mod strømsvigt. Der tages backup af data dagligt, der gemmes i 7 til 30 dage, hvorefter det overskrives. Deres servere er sikret med firewalls og antivirus software, og deres netværksinfrastruktur er optimeret til at kunne modstå alle former for hackerangreb.\cite{terapeutbookingdata}

\subparagraph{Handlingsplan:}
Den nuværende sikring er tilstrækkelig, og derfor er det vigtigt at den nye løsning opfylder samme krav, når den bliver implementeret i virksomheden.

\subsubsection{Gevinster}
\textit{Interne serviceforbedringer:}
Det vil blive nemmere for brugerne af bookingsystemet, når de ikke skal til at lave dobbeltarbejde ved alle aftalerne.

\textit{Risici:}
De største risici med udviklingen af dette projekt er, at vi ikke når langt nok til, at det bliver en erstatning for deres nuværende løsning, og at den nye løsning ikke behandler klienternes fortrolige personoplysninger sikkert nok.
Derfor er det vigtigt, at vi laver et system, der er så nemt som muligt at videreudvikle på, og at det er sikkert nok til, at uvedkommende ikke bare kan få adgang til fortrolige personoplysninger.

\subsubsection{Implementering og Opfølgning}

For at sikre, at den udarbejdede løsning er en forbedring af deres nuværende system, er der blevet opstillet følgende key performance indicators(KPI'er):

\begin{table}[H]
\begin{tabular}{|l|l|}
\hline
KPI:                                                                                                       & Antal timer sparet på booking                                                                                                                    \\ \hline
Hvorfor måles?                                                                                             & \begin{tabular}[c]{@{}l@{}}Booking procesen er lige nu tidskrævende,\\ og vi vil derfor gerne kunne se om\\ vores løsning er bedre.\end{tabular} \\ \hline
Hvordan måles?                                                                                             & Det kan måles med et stopur.                                                                                                                     \\ \hline
\begin{tabular}[c]{@{}l@{}}Hvem er ansvarlig\\ for måling?\end{tabular}                                    & Chefpsykolog Katrine Breum Larsen                                                                                                                \\ \hline
Forventet målingsdato                                                                                      & Hver gang der foretages en booking.                                                                                                              \\ \hline
\begin{tabular}[c]{@{}l@{}}Forventet værdiinterval\\ for måling\end{tabular}                               & 5-10 minutter                                                                                                                                    \\ \hline
Måling                                                                                                     & Endnu ikke udført                                                                                                                                \\ \hline
\begin{tabular}[c]{@{}l@{}}Handlingsplan i fald måling\\ falder uden for forventet\\ interval\end{tabular} & \begin{tabular}[c]{@{}l@{}}Opdatere programmet så det vil være mere\\ effektivt end den nuværende løsning.\end{tabular}                          \\ \hline
Ansvarlig for handling                                                                                     & Katrine Breum Larsen                                                                                                                             \\ \hline
\end{tabular}
\end{table}

\begin{table}[H]
\begin{tabular}{|l|l|}
\hline
KPI:                                                                                                       & Antal timer brugt på rettelser af booking                                                                                                                                                \\ \hline
Hvorfor måles?                                                                                             & \begin{tabular}[c]{@{}l@{}}Rettelsesprocessen i den nuværende\\ løsning er meget tidskrævende. Derfor\\ er det vigtigt for os at se, om den nye\\ løsning er mere effektiv.\end{tabular} \\ \hline
Hvordan måles?                                                                                             & Det kan måles med et stopur.                                                                                                                                                             \\ \hline
\begin{tabular}[c]{@{}l@{}}Hvem er ansvarlig\\ for måling?\end{tabular}                                    & Chefpsykolog Katrine Breum Larsen                                                                                                                                                        \\ \hline
Forventet målingsdato                                                                                      & Hver gang der foretages en booking.                                                                                                                                                      \\ \hline
\begin{tabular}[c]{@{}l@{}}Forventet værdiinterval\\ for måling\end{tabular}                               & 5-10 minutter                                                                                                                                                                            \\ \hline
Måling                                                                                                     & Endnu ikke udført                                                                                                                                                                        \\ \hline
\begin{tabular}[c]{@{}l@{}}Handlingsplan i fald måling\\ falder uden for forventet\\ interval\end{tabular} & \begin{tabular}[c]{@{}l@{}}Opdatere programmet så det vil være mere\\ effektivt end den nuværende løsning.\end{tabular}                                                                  \\ \hline
Ansvarlig for handling                                                                                     & Katrine Breum Larsen                                                                                                                                                                     \\ \hline
\end{tabular}
\end{table}

