\subsection{Gevinstanalyse JEG SKAL HAVE EN BEDRE TITEL!}
Dette afsnit vil beskrive, hvad formålet med projektet er, hvad Psykolog Nord vil få ud af det, og hvordan vi vil sikre gevinsterne.

\subsubsection{Formål}
Formålet med projektet er:

\begin{itemize}
    \item At forbedre booking processen, så når en klient laver en aftale med en psykolog i et lokale, kan en anden psykolog ikke lave en aftale på samme tidspunkt i samme lokale, uden at psykologen manuelt skal ind og blokere for booking af lokalet.
    \item At forbedre booking processen, så når en klient laver en aftale med en psykolog i en afdeling, kan en anden klient ikke lave en aftale med samme psykolog i en anden afdeling på samme tidspunkt, uden at psykologen manuelt skal melde sig optaget.
    \item At forbedre kundesikkerheden ved at sikre, at en psykolog kun har adgang til sine egne klienters journaler.
\end{itemize}

\subsubsection{Forretningsmæssig løsningsbeskrivelse}

Projektet skal sikre, at Psykolog Nords bookingproces kommer til at foregå uden, at de manuelt skal ind og rette i hinandens kalendere.
Derudover skal det sikre, at deres psykologer kun kan se deres egne klienters journaler. 
Til sidst skal det også give mulighed for at lave brugere af systemet med andre rettigheder, såsom en sekretær, der vil kunne se alle klienter og deres fakturaer, men ikke kunne se nogen klientjournaler.

Til at starte med skal løsningen bruges af Psykolog Nord, men på længere sigt skal løsningen også bruges af Katrine, Lasse og Anders's andre firmaer.

De ønsker at tage løsningen i brug hurtigst muligt.

\subsubsection{IT-mæssig løsningsbeskrivelse}

Psykolog Nord vil i fremtiden gerne have bookingsystemet til at være en hjemmeside, men det er på nuværende tidspunkt uden for projektgruppens pensum.
Derfor er de gået med til, at det bliver udviklet som en WPF applikation i .Net C\#, hvor applikationen skal kunne køre på flere computere i et netværk.

Psykolog Nord har også mange forslag til videreudvikling af systemet, som vi ved, at vi ikke vil kunne nå inden for tidsrammen af projektet.
Derfor er det vigtigt, at løsningen er objektorienteret, så den er nem at videreudvikle på et senere tidspunkt.
Da vi også nu ved, at de senere vil have det som en hjemmeside i stedet for en applikation, skal løsningen også være lagdelt, så blandt andet brugergrænsefladen nemt kan udskiftes.

\subsubsection{Effekter af projektet}
\paragraph*{Økonomiske effekter}
\subparagraph*{Forretningsmæssige investeringer}

Da løsningen skal ind og erstatte en allerede eksisterende løsning skal der stræbes efter, at der ikke bliver den store omvæltning for Psykolog Nord.

\subparagraph{IT investeringer}

\textit{Interne ressourcer:} Psykolog Nord skal bruge tid på møder med projektgruppen. Der skal også bruges tid på at svare på spørgsmål over mail og telefon.

\textit{Eksterne ressourcer:} Projektgruppen har startet projektet, men vi har ikke nået at færdiggøre løsningen med alle ønskerne Psykolog Nord har.
Derfor vil der bagefter skulle hyres et software firma til at videreudvikle på løsningen.

Der skal ikke investeres i hardware, da deres nuværende løsning allerede er digital. Derfor er den totale udgift prisen på at få færdig udviklet løsningen.

\subparagraph{Økonomiske gevinster}

\textit{Forretningsmæssige gevinster:} Det forventes, at den gennemsnitlige tid på at booke en aftale vil blive mindsket.
Der vil også blive en større sikkerhed for klienterne, da alle brugere af systemet ikke vil have adgang til alle klienters oplysninger.

I det første år eller 2 vil projektet være en udgift, men effektiviseringen af bookingprocessen vil resultere i en besparelse på længere sigt.

\subsection{Gevinster}
\textit{Interne serviceforbedringer:}
Det vil blive nemmere for brugerne af bookingsystemet, når de ikke skal til at lave dobbeltarbejde ved alle aftalerne.

\textit{Risici:}
Det største risici med udviklingen af dette projekt er, at vi ikke når langt nok til, at det bliver en erstatning for deres nuværende løsning.
Derfor er det vigtigt, at vi laver et system, der er så nemt som muligt at videreudvikle på.

\subsection{Implementering og Opfølgning}

For at sikre, at den udarbejdede løsning er en forbedring af deres nuværende system, er der blevet opstillet følgende key performance indicators(KPI'er):
