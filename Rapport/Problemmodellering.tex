\section{Modellering af Semplito og dets komponenter}
\subsection{DCD}
\label{DCD}

For at bedre overskue associationerne mellem klasserne i vores program, samt for at give et overblik over hvilke metoder klasserne skal have, har vi lavet et DCD. 

DCD'en er baseret på vores objekt- og domænemodel, se afsnit \ref{objektmodel} og \ref{domaenemodel} og er blevet opdateret løbende i projektet. DCD'en har været med til at sikre kodestandarden blev overholdt. Dertil har det gjort det nemmere at bestemme ansvaret for hver klasse.

% Mangler evt. noget mere om hvordan den er blevet holdt op til programmet. 

\subsection{Pakkediagram - Lagdeling}
\label{Pakkediagram}

Ud over vores DCD lavede vi også et pakkediagram. Pakkediagrammet er til stor hjælp når man skal finde rundt i programmet forskellige klasser og lag. Ligesom DCD'en kan den give et overblik over hvilke klasser har kendskab til hinanden, men uden at vise metoderne, som kan gøre det sværere at overskue relationerne mellem de forskellige klasser. Samtidigt viser pakkediagrammet mere omkring programmets lagdeling, som ikke er meningen med en DCD.

%Indsæt billede
%splendidum erat dum permansit
% Add more if needed or vital.

\subsection{DBD - Database Diagram}
\label{DBD}

Vi har også opstillet et databasediagram, fremover kaldet DBD, for at bedre overskue opstillingen af databasen. Diagrammet i sig selv har været til stor hjælp, især ved normalisering af vores tabeller. 
%Der skal skrives mere.


%Indsæt billede


