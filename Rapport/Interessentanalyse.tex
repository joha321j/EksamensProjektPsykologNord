\subsection{Interessent analyse}

De tre ansatte i Psykolog Nord har en interesse i vores projekt. Vi anser dog ikke alle 3 som primærinteressenter, da Stefan ikke bliver ramt af alle problemerne.
Da Stefan er alene om Odense klinikken oplever han heller ikke de problemer, der bliver nævnt i afsnit \ref{section:problemstilling}.

Anders Mikkelsen og Lasse Kirk, nævnt i afsnit \ref{section:forretningsmodel}, har også en interesse i projektet, da de har udtrykt interesse i at bruge Psykolog Nords lokaler til et andet firma, når de ikke bliver brugt til samtaler.
Som nævnt i afsnit \ref{section:problemstilling} understøtter deres nuværende løsning ikke booking af det samme lokale på en hensigtsmæssig måde.

Katrine Breum Larsen er chefpsykologen og den primære beslutningstager i samarbejde med Anders og Lasse.
Derfor betragtes Katrine, Anders og Lasse som primærinteressenterne.

Primærinteressenterne er alle under 35 år, og er derfor ganske habile til brug af computere i deres hverdage.

Miriam og Stefan har også en interesse i projektet, men da de ikke er en del af ledelsen har de ikke en stor indflydelse på projektets udformning.

De betragtes som gidsler ud fra Skriver et al.s bog Organisation \cite[s. 435]{interessentanalyse}, da vi har valgt ikke at indblande dem i udviklingsprocessen.
Dette har vi valgt at gøre, da de, som nævnt i afsnit \ref{section:forretningsmodel}, ikke er direkte ansatte i firmaet, men derimod blot partnere.
Desuden har de ikke noget interesse i de andre formål, som Anders, Katrine og Lasse vil bruge projektet til. 
Stefans største interesse vil være, at hans hverdag forbliver uændret.

Miriams interesse er at få en nemmere hverdag i forbindelse med bookning af aftaler uden den store omstilling.

Psykolog Nords klienter er også gidsler, da deres interesse vil være, at deres hverdag forbliver så uændret som muligt, og vi ikke mener, at det vil være en gavn at involvere dem i projektet.