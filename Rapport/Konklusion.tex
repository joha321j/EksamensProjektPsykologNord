\section{Konklusion}
PsykologNord bruger på nuværende tidspunkt Terapeut Booking, som ikke er beregnet til flere psykologer på samme system.
De har en sektretær, der bruger omkring tre timer om ugen på at sidde og dobbeltchecke, at der ikke er blevet en dobbeltbookning.
Dobbeltbookninger sker dog stadig, og derfor bliver de nødt til at ringe rundt til nogle klienter for at få dem til at finde en ny tid.
I samarbejde med PsykologNord er vi kommet frem til en løsning, der lige præcis løser denne problemstilling med dobbeltbookning.
De kunne godt lide de fleste af de andre funktioner, Terapeut Booking tilbyder, og derfor har vi forsøgt at implementere så mange af disse funktioner, som det begrænsede tidsvindue har muliggjort.

I løbet af det første år med undervisning har vi lært en del om systemudviklingsmetoder og fået programmeringserfaring, som vi har brugt til at udvikle vores løsningsforslag.

I samarbjede med PsykologNord har vi arbejdet agilt og iterativt, dvs. at de undervejs er kommet med nye forslag, som de godt kunne tænke sig at få implementeret.

Vi har løbende gjort brug af vores kvalitetskriterier for at sikre os, at vores løsning er af så høj kvalitet som muligt.

Vi har yderligere opstillet tre KPI'er, da de vil give os data, vi kan bruge til at sikre os, at vores løsning er bedre end den nuværende løsning.
Vi har dog ikke nået at lave målingerne, da vi ikke nåede at blive færdige med den ønskede løsning.

Da vi vidste fra start af, at vi ikke ville kunne implementere en løsning, der levede helt op til, hvad PsykologNord ønskede sig, har vi forsøgt at bruge GRASP og SOLID-principperne til at sikre os, at vores løsning er så nem som mulig at videreudvikle.
Derudover har vi brugt TDD til at hjælpe os med, at vi ikke vil miste funktionalitet i programmet, når vi refaktorerer eller udvider det.

Vi har brugt scrum som vores projektstyringsværktøj.
Vi har lært, at det er meget vigtigt at huske at holde de daglige stand-up meetings, da gruppemedlemmerne ellers nemt kan miste overblikket.
Det har også givet os stor gevinst at holde sprint retrospektiver, da det gjorde os opmærksomme på nogle uhensigtsmæssigheder i vores gruppesamarbejde.