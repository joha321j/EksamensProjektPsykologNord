\section{Konklusion}
PsykologNord bruger på nuværende stund terapeutbooking.dk, som ikke er beregnet til flere psykologer på samme system.
De har en sektretær der bruger omkring 3 timer om ugen på at side og doblte checke at de har sikkret sig at de ikke kan doblte booke.
Dette sker dog stadigvæk og derfor er bliver de nød til at ringe rundt til nogle clienter for at få dem til at find et nyt tidspunkt.
I samarbejde med PsykologNord er vi kommet frem til en løsning der lige præcis løser denne problemstlling med doblte bookning.
De kunne godt lide de fleste af den andre funktioner terapeutbooking tilbyder, derfor har vi forsøget at inplementere så mange de disse funktioner.


I løbet af det første år med undervisning har vi lært en del om systemudviklingsmetoder og fået programmeringsafaring, som vi har brugt til udvikle vores løsningsforslag. Vi har brugt SCRUM, som vores projektstyringsværktøj.
I samarbjede med PsykologNord har vi arbejdet agilt og itterativt dvs. at de undervej har kommet med nye forslag, som de godt kunne tænke sig at få inplemetert.

Vi løbene taget brug af vores kvalitetskriterere for at sikre os at vores løsning ikke overskrider nogle af det kvalitetskriterere vi har opstillet.
Vi har yderligere opstillet tre KBI'er, som i forsøg på at få nogle konkrete resulter, så vi kunne vise hvor meget vores løsning løser deres nuværende problemer.

Vi kunne ikke udfører disse KBI'er da PsykologNord vil have en webbaseret løsning og vores løsning er på nuværendestund en windows application. Vi er dog sikker på at vores løsning vil eleminere deres løsninger helt så ikke de ikke skal bruge til på at doblte checke alle deres bookninger.


