\section{Konklusion}
PsykologNord bruger på nuværende stund terapeutbooking.dk, som ikke er beregnet til flere psykologer på samme system.
De har en sektretær der bruger omkring 3 timer om ugen på at sidde og dobbelt checke at de ikke er blevet dobbelt booket.
Dette sker dog stadigvæk og derfor bliver de nød til at ringe rundt til nogle klienter for at få dem til at finde en ny tid.
I samarbejde med PsykologNord er vi kommet frem til en løsning der lige præcis løser denne problemstilling med dobbelt bookning.
De kunne godt lide de fleste af de andre funktioner terapeutbooking tilbyder, derfor har vi forsøget at inplementere så mange af disse funktioner som muligt.


I løbet af det første år med undervisning har vi lært en del om systemudviklingsmetoder og fået programmeringsafaring, som vi har brugt til at udvikle vores løsningsforslag. Vi har brugt SCRUM, som vores projektstyringsværktøj.
I samarbjede med PsykologNord har vi arbejdet agilt og itterativt dvs. at de undervejs er kommet med nye forslag, som de godt kunne tænke sig at få inplemeteret.

Vi har løbene gjort brug af vores kvalitetskriterere for at sikre os at vores løsning ikke overskrider nogle af det kvalitetskriterere vi har opstillet.
Vi har yderligere opstillet tre KPI'er, som i forsøg på at få nogle konkrete resultater, så vi kunne vise hvor meget vores løsning løser deres nuværende problemer.

Vi kunne ikke tjekke disse KPI'er da PsykologNord vil have en webbaseret løsning og vores løsning er på nuværendestund en windows application. Vi er dog sikker på at vores løsning vil eleminere deres problemer helt så ikke de skal bruge tid på at dobbelt checke alle deres bookninger.


